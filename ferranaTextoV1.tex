\documentclass[
	article,			% indica que é um artigo acadêmico
	11pt,				% tamanho da fonte
	oneside,			% para impressão apenas no recto. Oposto a twoside
	a4paper,			% tamanho do papel. 
	english,			% idioma adicional para hifenização
	brazil,				% o último idioma é o principal do documento
	sumario=tradicional
	]{abntex2}

\usepackage{lmodern}			% Usa a fonte Latin Modern
\usepackage[T1]{fontenc}		% Selecao de codigos de fonte.
\usepackage[utf8]{inputenc}		% Codificacao do documento 
\usepackage{indentfirst}		% Indenta o primeiro parágrafo de cada seção.
\usepackage{nomencl} 			% Lista de simbolos
\usepackage{color}				% Controle das cores
\usepackage{graphicx}			% Inclusão de gráficos
\usepackage{microtype} 			% para melhorias de justificação
\usepackage{lipsum}				% para geração de dummy text
\usepackage[alf]{abntex2cite}	% Citações padrão ABNT


\titulo{ferrana - Modelo Preditor para Eleições Proporcionais no Brasil}
\autor{Gabriel  Melo e Matheus Pestana\thanks{Mestrandos em Ciência Política pelo IESP-UERJ.  E-mail: gabriel.melo@iesp.uerj.br e matheus.pestana@iesp.uerj.br}}
\local{Brasil}
\data{2018}

\definecolor{blue}{RGB}{41,5,195}


\makeatletter
\hypersetup{
     	%pagebackref=true,
		pdftitle={\@title}, 
		pdfauthor={\@author},
    	pdfsubject={Modelo de artigo científico com abnTeX2},
	    pdfcreator={LaTeX with abnTeX2},
		pdfkeywords={abnt}{latex}{abntex}{abntex2}{atigo científico}, 
		colorlinks=true,       		% false: boxed links; true: colored links
    	linkcolor=blue,          	% color of internal links
    	citecolor=blue,        		% color of links to bibliography
    	filecolor=magenta,      		% color of file links
		urlcolor=blue,
		bookmarksdepth=4
}
\makeatother

\makeindex

\setlrmarginsandblock{3cm}{3cm}{*}
\setulmarginsandblock{3cm}{3cm}{*}
\checkandfixthelayout

\setlength{\parindent}{1.3cm}

\setlength{\parskip}{0.2cm}  


\SingleSpacing

\begin{document}

\selectlanguage{brazil}

\frenchspacing 

\maketitle


\textual

\section*{Introdução}

O objetivo do trabalho consiste em construir um modelo estrutural capaz de prever a quantidade de votos e cadeiras que determinado partido obterá nas eleições proporcionais para a Câmara dos Deputados na eleições de 2018. A justificativa para a construção de um modelo estrutural é a ausência de pesquisas de opinião pública relativas à intenção de voto para a disputa de cadeiras na Câmara dos Deputados, que permitiriam um modelo agregado ou sintético. Observando as considerações de \citeonline{lewis2015forecasting} a partir do contexto europeu, no entanto, modelos estruturais ainda são considerados mais precisos.

A coleta de dados se deu através do pacote ``electionsBR'' disponível no repositório da linguagem de programação R, que funciona como interface para o download direto do site do Tribunal Superior Eleitoral. Contamos, todavia, com dificuldades na captação dos dados para as eleições de 1945 a 1994, sendo necessárias modificações no código-fonte do pacote, a serem disponibilizadas posteriormente à comunidade. A dificuldade inicial no tratamento de nossa base de dados se deu pelo fato destes estarem codificados de maneira diferente em eleições anteriores a 1998, utilizando, por exemplo, unidades de análise geográficas distintas.

As variáveis que optamos por utilizar por ora em nosso modelo estrutural são: partido, estado, ano da eleição, quantidade de votos nominais, quantidade de votos de legenda, quantidades de votos totais, quantidade de cadeiras, desemprego no ano da eleição, IPCA agregado até setembro do ano da eleição e taxa de crescimento do PIB do ano anterior, variáveis macroeconômicas que satisfazem os requisitos para a construção de um modelo estrutural.

\section*{Primeiros passos}

A figura \ref{fig:cadeiras-partido}  apresenta a quantidade de cadeiras por partido desde as eleições de 1994. Selecionamos os quatro maiores partidos brasileiros do período posterior à redemocratização. Observamos que apenas PFL/DEM, PMDB e PSDB foram capazes ao longo deste período de conquistar 100 ou mais cadeiras, algo esperado em sistemas eleitorais proporcionais. Observa-se também que tais partidos perdem espaço com a ascensão do número de cadeiras obtidas por outros partidos. 

\begin{figure}
	\centering
	\includegraphics[scale=0.5]{CadeirasPartido.png}
	\caption{Cadeiras por partido desde 1994.}
	\label{fig:cadeiras-partido}
\end{figure}

Na figura \ref{fig:votos-nominais-partido} observamos a evolução dos votos em candidaturas pessoais ao longo do tempo. É possível observar a ausência de dados para o ano de 2002, o que se deve a um problema com a interface do TSE a ser corrigido para a próxima versão. Com os dados disponíveis, todavia, podemos verificar nos últimos anos a fragmentação da Câmara dos Deputados a partir da emergência de novos partidos, quadro observado em menor escala no Rio de Janeiro, como observado na figura \ref{fig:votos-partido-rio}, que mostra a ascensão do PMDB. 

\begin{figure}
	\centering
	\includegraphics[scale=0.5]{RioVotosPartido.png}
	\caption{Votos nominais no estado do Rio de Janeiro desde 1994.}
	\label{fig:votos-partido-rio}
\end{figure}

A figura \ref{fig:votos-leg} demonstra a dominância de PT e PSDB como maiores receptores de voto de legenda, quadro que se mantém relativamente estável ao longo do tempo, com notável declínio do PSDB em 2002 e recuperação nas eleições seguintes, eventualmente superando o PT em 2014.

\begin{figure}
	\centering
	\includegraphics[scale=0.5]{VotosLegPartido.png}
	\caption{Votos em legenda, por partido, desde 1994.}
	\label{fig:votos-leg}
\end{figure}

\begin{figure}
	\centering
	\includegraphics[scale=0.5]{VotosNomPartido.png}
	\caption{Votos nominais por partido desde 1994.}
	\label{fig:votos-nominais-partido}
\end{figure}

\section*{Próximos passos}

A etapa seguinte consistirá na junção do banco de dados com o período anterior às eleições de 1994. Adicionaremos também as variáveis econômicas para a construção do modelo preditor.

\bibliography{referencias.bib}

\end{document}
